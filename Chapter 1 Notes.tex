\documentclass[12pt,a4paper]{article}

% Packages
\usepackage[utf8]{inputenc}
\usepackage[margin=1in]{geometry}
\usepackage{amsmath, amssymb, amsthm}
\usepackage{graphicx}
\usepackage{float}
\usepackage{hyperref}
\usepackage{xcolor}
\usepackage{fancyhdr}
\usepackage{enumitem}

% Header and Footer
\pagestyle{fancy}
\fancyhf{}
\fancyhead[L]{Neuroscience Notes}
\fancyhead[R]{\thepage}

% Custom Commands
\newcommand{\concept}[1]{\subsection*{\textcolor{blue}{Concept: #1}}}
\newcommand{\definition}[1]{\subsubsection*{\textcolor{teal}{Definition: #1}}}
\newcommand{\example}[1]{\paragraph{\textcolor{purple}{Example:}} #1}
\newcommand{\summary}{\section*{\textcolor{red}{Summary}}}
\newenvironment{keypoints}{\paragraph{\textbf{Key Points:}}\begin{itemize}[label=--]}{\end{itemize}}

% Title
\title{\textbf{Neuroscience Notes}}
\author{}
\date{}

\begin{document}
	
	\maketitle
	\tableofcontents
	\newpage
	
	\section*{Biopsychology as a Neuroscience}
	Write an overview of the topic here.
	
	\section{Biopsychology as a Neuroscience}
	\concept{What is Biopsychology?}
	\definition{The scientific study of \textbf{biology of} behavior.}
	\begin{itemize}
		\item Biopsychology is also referred to as physiological psychology, psychobiology, behavioral biology, or behavioral neuroscience.
		\item Whenever we talk about biopsych, there must a biological approach of biological aspect. 
	\end{itemize}
	\example{Reaction to a drug, specific symptom of a mental disorder, change in perception.}
	\begin{keypoints}
		\item Figuring out: \textbf{What is the biological mechanism of whatever it is that's happening?}.
	\end{keypoints}
	\concept{Why studying biopsychology?}
	\definition{Allows us to examine the activity of the nervous system and how it underlies our mental experiences and behavior.}
	\begin{itemize}
		\item How the activities, changes, in the nervous system affect our emotion, our mind, our behavior, how we perceive things on a day to day basis. 
	\end{itemize}
	
	\concept{What do biopsychologist study?}
	\definition{There are various specializations biopsychologist can adopt. Some of these specializations include:}
	\begin{itemize}
		\item Psychology 
		\subitem Studying human reaction, human emotion in response to a certain stimulus.
		\item Biology
		\subitem Studying a specific set of proteins, or a specific molecule present in the brain. 
		\item Neurology 
		\item Psychiatry
		\subitem Neurology and psychiatry based is when the doctors prescribed drug, they asked: "How does this drug affect my patient's behavior?"
		\item Physiology 
		\item Engineering 
		\subitem Neural-engineer, study CBI, develop prosthetics. 
		\subitem Collaborate with biologist and biopyschologist to solve any issue that the engineer have and improve their products. 
		\subsubitem Collaboration between these specializations are vital in furthering our knowledge of the brain and behavior. 
	\end{itemize}
	\concept{What do biopsychologist study?}
	\definition{Biopsychologists typically receive training in one or more of the below areas:}
	\begin{itemize}
		\item Neurobiology: The study of the underlying biological bases of the nervous system. 
		\subitem Thinking about how the biological aspect of the human body, like proteins, affect human behavior. 
		\item Neurochemistry: The study of chemical compounds that modulate the nervous system. 
		\subitem Study neurotransmitters: serotonin, dopamine, etc. 
		\item Neurophysiology: The study of nervous system function. 
		\item Neuroanatomy: The study of nervous system structure. 
		\item Neuropathology: The study of nervous system disorders. 
		\item Neuropharmacology: The study of psychotropic drugs. 
	\end{itemize}
	
	\section*{Biopsychology Research}
	Write an overview of the topic here.
		
	\section{Biopsychology Research}
	\concept{How do we approach research in biopsychology?}
	\definition{Biopsychology consists of methodologies involving the following:.}
	\begin{itemize}
		\item Human and non-human subjects.
		\subitem Human subject
		\subsubitem Advantages to using human subjects...
		\begin{itemize}
			\item Humans are capable of following directions. 
			\item Humans can report subjective experiences, i.e. feelings, mood, thoughts, etc. 
		\end{itemize}	
		\subsubitem However, there are at times ethical and practical constraints with humans, so animal research becomes necessary.
		\subitem Non-human subjects 
		\subsubitem In biopsychology, rats and mice are most used as non-human subjects. 
		\subsubitem Advantages to using rodents in research: 
		\begin{itemize}
			\item Simpler models with similar brain structures, genes, and developmental processes. 
			\example{In human, at the shift from the age of three to four, there is a proteins that helps with memory consolidation. This also happen in rodents at around two weeks of age.}
			\subitem Results in these rodents, there non-human beings are generalizable to human beings. 
			\item Allow for a comparative approach, i.e., we can study biological processes by comparing results of different species. 
			\subsection{Ethics in Animal Research}
			\begin{itemize}
				\item Animal research has shown us a lot about how our brains work. 
				\item But, there is an apparent issue here, which is simultaneously being able to increase our knowledge of a phenomenon while also decreasing discomfort and distress in animals. 
			\end{itemize}
			\subsection{Oversight of Animal Research}
			There will always be the question of if animal experimentation constitutes animal cruelty. 
			\begin{itemize}
				\item However, upon approval for an animal-based study, researchers must prove they have explored all other options before resorting to the utilization of animals. If there is no other way to conduct the study, permission to use animals will be granted. 
				\item Speed at which diseases are studied vs. animal experimentation.
				\item Most institutions have very strict regulations regarding animal use...
				\subitem IACUC (Institutional Animal Care and Use Committee)
			\end{itemize}
		\end{itemize} 
		\item Experiments and non-experiments.
		\subitem Experiments
		\definition{An \textbf{experiment} is a method used to determine whether a change in one variable \textbf{causes} a change in another variable...}
		\subsubitem Required characteristics of an experiments include:
		\begin{itemize}
			\item Random assignment 
			\item Avoidance of confounding variables
			\subitem Make as many steps as possible to mitigate these variables.  
			\item Independent variable (treatment manipulated by experimenter)
			\item Dependent variable (variable measured by the experimenter)
		\end{itemize} 
		\subitem Non-experiments
		\subsubitem Although an experiment is typically the ideal way to approach a study, there are some cases where this is not feasible. Therefore, one of the two approaches below may be used:
		\begin{enumerate}
			\item A \textbf{quasi-experiment} is a method used study the effects of one variable on another when random assignment to treatment groups is not possible. 
			\example{Patients with a certain psychiatric condition}
			\item A \textbf{case study} is an in-depth focus on a single case or subject.
			\example{Amnesiac patient H.M. and usually not generalizable}
		\end{enumerate}
		\item Basic(pure) and applied research.
		\definition{\textbf{Basic (pure) research} is pursued when the purpose is solely to acquire knowledge.} 
		\subitem Motivated by the curiosity of the researcher. 
		\subsubitem To understand something.
		\definition{\textbf{Applied research} is pursued when the intention is to directly benefit mankind.}
		\subitem Many studies/research programs have elements of both.  
		\subsubitem To solve problems, to come up with solutions. 
	\end{itemize}
	
	\section{The subdisciplines of Biopsychology}
	
	There are six major subdisciplines of biopsychology...
	\begin{itemize}
		\item Physiological psychology
		\subitem Biopsychologist that based in physiology, how nervous system or our central nervous function, that affect our behavior. 
		\item Psychopharmacology
		\subitem Studies drugs, see how drugs affect people behavior. 
		\item Neuropsychology
		\subitem Centered around studying different legion in the brain affect how people behave. 
		\item Psychophysiology
		\subitem How are things are working in the nervous system, in terms of our mood, our feeling, and our behavior. 
		\item Cognitive neuroscience
		\subitem How is our brain affecting how we are perceiving things, processing things. 
		\item Comparative psychology
		\subitem Comparing the behavior between different species and find similarities and differences. 
	\end{itemize}
	
	\concept{Major subdisciplines in details}
	
	\subsection{Physiological Psychology}
	
	\definition{\textbf{Physiological Psychology} is the study of the neural mechanisms of behavior and perception.}
	
	\begin{itemize}
		\item Direct manipulation and recording of the brain. 
		\item Commonly basic research, unless experimentation is aimed at developing a new apparatus. 
	\end{itemize}
	
	\example{EEG, electrophysiological study}
	
	\subsection{Psychopharmacology}
	
	\definition{\textbf{Psychopharmacology} is the study of how drugs affect our behavior.}
	
	\begin{itemize}
		\item Both humans and non-humans.
		\item Both basic and applied. 
	\end{itemize}
	
	\subsection{Neuropsychology}
	
	\definition{\textbf{Neuropsychology} is the study of the psychological effects of brain damage.}
	
	\begin{itemize}
		\item Neuropsychological testing facilitates diagnoses and establishment of treatment. 
		\item Applied research in human patients. 
	\end{itemize}
	
	\subsection{Psychophysiology}
	
	\definition{\textbf{Psychophysiology} is the study of the relationships between physiological activity and psychological processes.}
	
	\begin{itemize}
		\item Record physiological activity for the aim of establishing a relationship between physiology and behavior. 
		\item Non-invasive measures in human subjects. 
		\item Basic and/or applied. 
	\end{itemize}
	
	\example{Hypothalamic Pituitary Adrenal axis - HPA axis}
	
	\subsection{Cognitive neuroscience}
	
	\definition{\textbf{Cognitive Neuroscience} is the study of the neural bases of cognition.}
	
	\begin{itemize}
		\item Comprised of studying (subfield of neuroscience):
		\begin{itemize}
			\item How we sense things
			\item How we perceive
			\item How we recognize
			\item How we attend to things
			\item Learning and memory
			\item Language
			\item Decision making, and motor control. 
		\end{itemize}
		\item Usually non-invasive in human studies. 
		\item Both basic and/or applied. 
	\end{itemize}
	
	\subsection{Comparative psychology}
	
	\definition{\textbf{Comparative psychology} is the study of similarities and differences in the behaviors of different species.}
	
	\begin{itemize}
		\item Lab setting or natural setting (ethological or ecological? research)
		\item Basic research
	\end{itemize}
	
	Pictures in different subdisciplines from the textbook. 
	
	\concept{Contributions of Biopsychology}
	
	Constant overlap of the above disciplines allows us to...
	
	\begin{itemize}
		\item Satisfy our curiosity about the brain-behavior relationship.
		\item Increase our understanding of brain disorders and brain disorder treatment. 
	\end{itemize}
	
	\summary
	Summarize key findings or notes about neural signaling.
	
	\section{Figures and Diagrams}

	
	
	\section{References}
	Include all references used here in standard citation format.
	
\end{document}
