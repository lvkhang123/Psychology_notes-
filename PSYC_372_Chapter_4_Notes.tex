\documentclass[12pt,a4paper]{article}

% Packages
\usepackage[utf8]{inputenc}
\usepackage[margin=1in]{geometry}
\usepackage{amsmath, amssymb, amsthm}
\usepackage{siunitx}
\usepackage{graphicx}
\usepackage{float}
\usepackage{hyperref}
\usepackage{xcolor}
\usepackage{fancyhdr}
\usepackage{enumitem}
\usepackage{mhchem}

% Header and Footer
\pagestyle{fancy}
\fancyhf{}
\fancyhead[L]{Neuroscience Notes}
\fancyhead[R]{\thepage}

% Custom Commands
\newcommand{\concept}[1]{\subsection*{\textcolor{blue}{Concept: #1}}}
\newcommand{\definition}[1]{\subsubsection*{\textcolor{teal}{Definition: #1}}}
\newcommand{\example}[1]{\paragraph{\textcolor{purple}{Example:}} #1}
\newcommand{\summary}{\section*{\textcolor{red}{Summary}}}
\newenvironment{keypoints}{\paragraph{\textbf{Key Points:}}\begin{itemize}[label=--]}{\end{itemize}}

% Title
\title{\textbf{Neuroscience Notes}}
\author{}
\date{}

\begin{document}
	
	\maketitle
	\tableofcontents
	\newpage
	
	\section*{The Resting Membrane Potential}
	Write an overview of the topic here.
	
	\section{Basics of Neuronal Communication}
	\begin{enumerate}
		\item Individual neurons are activated by \textbf{electrical signals}. 
		\item Neurons communicate with other neurons via 
		\textbf{chemical signals}.
	\end{enumerate}
	
	There are both electrical and chemical properties of cell membranes that must be understood for the conceptualization of neuronal transmission. 
	
	\section{The Cell Membrane}
	\begin{itemize}
		\item The wall of the cell that separates the intracellular space (inside of the cell) from the extracellular space (outside of the cell, i.e., extracellular fluid).
		\item Phospholipid bilayer with embedded proteins/sugars. 
	\end{itemize}
	
	\section{Selective Permeability}
	
	\begin{itemize}
		\item Most chemicals cannot pass through the cell membrane freely.
		\subitem Flow and allowance of molecules controlled by specific protein channels embedded within the membrane. 
		\subitem The opening and closing of these channels is dependent on electrical dynamics of the cell itself vs. the extracellular fluid. 
	\end{itemize}
	
	\section{Membrane Potential}
	
	Think of neurons like batteries...
	\begin{itemize}
		\item There is a difference in \textbf{voltage, or electrical charge,} between the intracellular and extracellular space. 
		\item This difference is called the membrane potential. 
	\end{itemize}
	\subsection{How was the membrane potential discovered?}

	Scientists used the axon of a giant squid to examine the membrane potential:
	\begin{itemize}
		\item Used \textbf{microelectrodes}: extremely fine electrodes used for cellular recordings. 
		\item How this was done:
		\begin{enumerate}
			\item Isolated an axon and placed it in salt water. 
			\item Placed one electrode in the axon, and one in the salt water. 
			\item Measured the \textbf{difference in charge} between the inside and outside of the axon. 
		\end{enumerate}
	\end{itemize}
	
	\section{The resting potential}
	
	The \textbf{resting potential}: the membrane potential of a neuron at rest (i.e., not sending a signal)
	\begin{itemize}
		\item The resting membrane potential of a neuron is typically at \textbf{-70 millivolts (-70mV)}
	\end{itemize}
	
	\subsection{Neuron polarization}
	\begin{itemize}
		\item When a neuron is at rest, the inside of the neuron is negatively charged in relation to the extracellular fluid. 
		\item In this case, we say that the membrane is \textbf{polarized}.
	\end{itemize}
	
	\section{Why is there a "resting potential"?}
	
	\begin{itemize}
		\item There is variation in the distribution of different \textbf{ions} inside and outside of the membrane. 
	\end{itemize}
	
	\subsection{What are ions?}
	\begin{itemize}
		\item Ions are \textbf{charged particles} that make up the salts in neural tissue and extracellular fluid. 
		\item There are two basic types. 
		\begin{enumerate}
			\item Cations: all positively charged ions. 
			\item Anions: all negatively charged ions. 
		\end{enumerate}
		\item Important ions for basic neuronal transmission are:
		\subitem Sodium (\ce{Na+})
		\subitem Potassium (\ce{K+})
		\subitem Chloride (\ce{Cl-})
		\subitem Calcium (\ce{Ca^2+})
		\subitem Organic anions (-)
	\end{itemize}
	\subsection{Ion Distribution}
	
	There is variation in the distribution of different ions inside and outside of the cell when the neuron is at rest...
	\begin{itemize}
		\item There is more \ce{Na+}, \ce{Cl-}, and \ce{Ca^2+} ions outside of the cell in the extracellular fluid. 
		\item There are more \ce{K+} ions inside of the cell. 
		\item There are also organic anions (-) "stuck" inside of  the cell (too large to move out).
	\end{itemize}
	
	\subsection{Ion Channels}
	
	Ions can only move in and out of the cell via \textbf{ion channels} embedded in the cell membrane.
	\begin{itemize}
		\item Each ion channel is selective to only one ion channel (i.e., sodium only moves through sodium ion channels)
		\item These channels are \textbf{voltage-gated}
		\subitem Opening and closing is dependent on the value of the membrane potential. 
		\item Most of these \textbf{ion channels} are \textbf{closed at rest}.
	\end{itemize}
	
	\subsection{Sodium-Potassium Pumps}
	
	Sodium-potassium pumps also contribute to the uneven distribution of ions at rest...
	
	\begin{itemize}
		\item These are protein complexes that are also embedded within the cell membrane, but are different from typical ion channels. 
		\item Utilize \textbf{active transport mechanisms} that rely on \textbf{ATP} (adenosine triphosphate, provides energy to support cellular processes) for energy. 
		\item Transports \textbf{three \ce{Na+} ions out} of the cell for every \textbf{two \ce{K+} ions} it draws \textbf{in} > brings neuron back to resting state following activation. 
	\end{itemize} 
	
	\section*{The Action Potential}
	Write an overview of the topic here.
	
	\section{Action Potentials}
	
	Action potentials are electrical impulses that are triggered when neurons are stimulated: 
	\begin{itemize}
		\item Action potentials, or APs, occur due to ions moving through voltage-gated channels in the cell membrane.
	\end{itemize}
	
	\subsection{Steps of an Action Potential}
	
	\begin{enumerate}
		\item Resting state 
		\item Depolarizing phase
		\item Repolarizing phase
		\item Hyperpolarizing phase
	\end{enumerate}
	
	\subsubsection{Resting Phase}
	
	The state of a neuron at rest, or not signaling. 
	
	\subsubsection{Depolarizing Phase}
	
	\begin{itemize}
		\item A threshold of excitation is reached at the cell membrane. 
		\item Voltage-gated \ce{Na+} ion channels open, \ce{Na+} ions rush into the cell (remember, at rest, \ce{Na+} is highly concentrated on the outside of the cell).
		\subitem The "Peak" of the AP
		\subsubitem The action potential quickly reaches its peak; \ce{Na+} ion channels become inactivated and close, this means no more \ce{Na+} is able to enter the cell from the outside. 
	\end{itemize}
	
	\subsubsection{Repolarizing Phase}
	\begin{itemize}
		\item \ce{K+} ion channels open; \ce{K+} ions flow out of the neuron (remember, at rest, \ce{K+} is highly concentrated on the inside of  the cell).
		\subitem Cell membrane begins to repolarize (return to the resting potential value). 
	\end{itemize}
	
	\subsubsection{Hyperpolarizing Phase}
	
	\begin{itemize}
		\item \textbf{Hyperpolarization}: Increase in negative charge relative to the resting potential. 
		\item \ce{K+} channels stay open, \ce{K+} ions continue to leave the cell.
		\subitem Causes cell membrane to "understood" the normal resting potential value. 
	\end{itemize}
	
	\subsubsection{Return to Resting State}
	
	\begin{itemize}
		\item \ce{K+} ion channels close
		\subitem Causes the membrane potential to return to the resting value. 
		\subitem Sodium-potassium pumps help to restore the resting potential. 
	\end{itemize}
	
	\subsection{Refractory Periods}
	
	\textbf{Absolute refractory period}
	
	\begin{itemize}
		\item 1-2 ms after an AP has been fired, during an absolute refractory period, it is impossible for the cell to fire another AP. 
	\end{itemize}
	
	\textbf{Relative refractory period}
	
	\begin{itemize}
		\item For a few more ms, an AP can only be fired with a higher-than-normal level of stimulation. 
	\end{itemize}
	
	\subsection{The All-or-None Law}
	
	\begin{itemize}
		\item Once an action potential is triggered, the impulse travels the length of the axon without decreasing the strength. 
		\item This will be the maximal response for the maintenance of the same amplitude. 
	\end{itemize}
	
	\subsubsection{How is the all-or-none law possible?}
	
	\begin{itemize}
		\item \ce{Na+} ion channels are heavily concentrated at the \textbf{Nodes of Ranvier}.
		\item Action potentials "jumps" from node to node. 
		\subitem I.e., \textbf{Saltatory conduction}
		\item The \textbf{myelin sheath} thus speeds up the conduction of action potentials.
		\item \textbf{Multiple sclerosis}: autoimmune disease that results in destruction of the myelin sheaths of neurons. 
		\subitem Results in slower transmission of action potentials. 
	\end{itemize}
	
	\section{Postsynaptic Potentials}
	
	\subsection{Pre vs. Postsynatpic Neurons}
	
	\begin{itemize}
		\item When a neuron fires an AP, a chemical signal is released from its terminal buttons. 
		\item The signal then diffuses across the synapse. 
		\subitem Synapse: the gap and point of communication between two neurons. 
		\item The presynaptic neuron releases the signal, the postsynaptic neuron receives it. 
	\end{itemize}
	
	\subsection{Postsynaptic potentials}
	
	\begin{itemize}
		\item The chemical signal binds to receptors in the postsynaptic neuron's cell membrane. 
		\item The signal can have one of two effects on the postsynaptic membrane. 
		\subitem Depolarization (increases the membrane potential)
		\subitem Hyperpolarization (decreases the membrane potential)
	\end{itemize}
	
	\subsection{EPSPs and IPSPs}
	
	\begin{itemize}
		\item Depolarizations are called excitatory postsynaptic potentials (EPSPs).
		\subitem These increase the likelihood that the postsynaptic neuron will fire an action potential. 
		\item Hyperpolarizations are inhibitory postsynaptic potentials (IPSPs).
		\subitem These decrease the likelihood that the postsynaptic neuron will fire an action potential. 
	\end{itemize}
	
	\subsection{Summation of postsynaptic potentials}
	
	\begin{itemize}
		\item Both EPSPs and IPSPs are graded responses
		\subitem Weak signals elicit weak PSPs. 
		\subitem Strong signals elicit strong PSPs. 
		\item Multiple PSPs are integrated and summed together to produce an overall effect to a postsynaptic neuron. 
	\end{itemize}
	
	\subsection{Integration of PSPs}
	
	\begin{itemize}
		\item Spatial summation: PSPs produced simultaneously on different parts of the postsynaptic neuron. 
		\item Temporal summation PSPs produced in rapid sequence at the same part of the postsynaptic neuron. 
	\end{itemize}
	
	\section{The Synapse and Neurotransmitters}
	
	\subsection{The Synapse}
	\definition{The Synapse} The gap between neurons where chemical communication happens. 
	\subsection{Types of synapses}
	\begin{itemize}
		\item Axodendritic
		\subitem Terminal buttons of the presynaptic neuron synapse onto dendrites of the postsynaptic neuron. 
		\item Axosomatic
		\subitem Terminal buttons of the presynaptic neuron synapse onto the soma of the postsynaptic neuron. 
		\item Axoaxonic
		\subitem Terminal buttons of the presynaptic neuron synapse onto the axon of the postsynaptic neuron. 
	\end{itemize}
	
	\subsection{Anatomy of a Synapse}
	\begin{itemize}
		\item Presynaptic membrane: cell membrane at the ends of the terminal buttons of the presynaptic cell. 
		\subitem Site from which chemical signals called neurotransmitter are released into the synapse. 
		\item Neurotransmitters are stored in synaptic vesicles until release. 
		\item Postsynaptic membrane: cell membrane of the cell receiving the signal, or the postsynaptic neuron. 
		\subitem Contain receptors, which are binding sites for chemicals released from the presynaptic cell. 
	\end{itemize}
	
	\section{Neurotransmitters}
	
	Neurotransmitters are the chemical signals released across the synapse
	
	There are several classes:
	\begin{enumerate}
		\item Amino acids
		\item Monoamines
		\item Acetylcholine 
		\item Neuropeptides
		\item "Unconventional" neurotransmitters
	\end{enumerate}
	
	\subsection{Amino Acid Neurotransmitters}
	
	\begin{itemize}
		\item Amino acids: The building blocks of proteins.
		\item We have excitatory and inhibitory neurotransmitters...
		\subitem Excitatory: \textbf{glutamate}, aspartate.
		\subitem Inhibitory: GABA, glycine. 
	\end{itemize}
	
	\subsection{Monoamine Neurotransmitters}
	
	\begin{itemize}
		\item These are called "monoamines" because they are synthesized from a single amino acid. 
		\subitem Aromatic amino acids such as phenylalanine, tryptophan (indolamines), and tyrosine (catecholamines)
		\item Within this group, we have catecholamines and indolamines.
		\subitem Catecholamines: dopamine, norepinephrine, epinephrine. 
		\subitem Indolamines: Serotonin. 
	\end{itemize}
	
	\subsection{Acetylcholine}
	
	\begin{itemize}
		\item Choline molecule with an added acetyl group.
		\item Widespread in the CNS and at neuromuscular junctions. 
	\end{itemize}
	
	\subsection{Neuropeptides}
	
	\begin{itemize}
		\item Long chain of amino acids
		\subitem Pituitary peptides: Oxytocin.
		\subitem Hypothalamic peptides: Corticotropin-releasing hormone (CRH)
		\subitem Brain-gut peptides: cholecystokinin.
		\subitem Opioid peptides: substance P. 
	\end{itemize}
	
	\section*{Synaptic Transmission}
	Write an overview of the topic here.
	
	\section{Synaptic Transmission}
	
	\begin{itemize}
		\item APs arrive at the terminal buttons of a neuron and trigger the release of neurotransmitter molecules into the synapse. 
		\item 4 major events: 
		\begin{enumerate}
			\item Neurotransmitter synthesis and storage
			\item Neurotransmitter release.
			\item Activation of postsynaptic receptors.
			\item Neurotransmitter inactivation and reuptake. 
		\end{enumerate}
	\end{itemize}
	
	\subsection{Neurotransmitter synthesis and storage}
	
	\begin{itemize}
		\item Neurotransmitters are synthesized in the cytoplasm of the cell body or terminal buttons. 
		\item These synthesized NTs are then packaged into vesicles. 
		\item Vesicles are stored in clusters next to the presynaptic membrane. 
		\item A neuron can synthesize and release more than one kind of neurotransmitter (i.e., \textbf{co-existence})
		\example{GABA and a neuropeptide}
	\end{itemize}
	
	\subsection{Neurotransmitter Release}
	
	\begin{itemize}
		\item Exocytosis: the process of neurotransmitter molecule release from synaptic vesciles and into the synaptic cleft. 
		\begin{enumerate}
			\item An AP is transmitted down the axon of the presynaptic neuron and reaches the terminal buttons. 
			\item Depolarization causes voltage-gated calcium ion channels to open on the presynaptic membrane. 
			\item Calcium ions enter the presynaptic membrane. 
			\item Synaptic vesicles fuse with the presynaptic membrane and empty their contents into the synaptic cleft. 
		\end{enumerate}
	\end{itemize}
	
	\subsection{Activation of Postsynaptic Receptors}
	
	\begin{itemize}
		\item Released NTs attach to \textbf{receptors} on the postsynaptic membrane (usually on the dendrites).
		\item \textbf{Ligand}: any molecule that binds to a receptor (NTs are ligands).
		\item Each receptor contains binding sites for particular types on NTs. 
		\subitem A NT can only influence cells that have receptors for that specific NT.  
	\end{itemize}
	
	\subsubsection{Receptor Subtypes}
	
	\begin{itemize}
		\item Most neurotransmitters can bind to more than one receptor subtype. 
		\subitem \example{GABA\textsubscript{A} and GABA\textsubscript{B}, dopamine D\textsubscript{1-5}, GluN2\textsubscript{A-D}.
		}
		\item Receptor subtypes typically trigger different types of responses in the postsynaptic neuron...
		\subitem Dopamine binds to D1 > excitatory effect. 
		\subitem Dopamine binds to D2 > inhibitory effect. 
	\end{itemize}
	
	\subsection{Neurotransmitter Inactivation and Uptake}
	
	\begin{itemize}
		\item Neurotransmitter signals must be terminated after release and binding to postsynaptic receptors. 
		\item Via two mechanisms...
		\begin{enumerate}
			\item Reuptake
			\item Enzymatic degradation
		\end{enumerate}
	\end{itemize}
	
	\subsubsection{Neurotransmitter Reuptake}
	
	\begin{itemize}
		\item \textbf{Transporters}: proteins in the cell membrane that bring neurotransmitter molecules back into the presynaptic neuron for repackaging and re-use. 
		\subitem \example{Serotonin (5-HT) is taken back up by serotonin transporters (SERTs)}
	\end{itemize}
	
	\subsubsection{Enzymatic Degradation of NTs}
	\begin{itemize}
		\item Enzymes break neurotransmitter molecules down into inactive chemicals that eventually wash away. 
		\subitem \example{Acetylcholinesterase (AChE) breaks down acetylcholine (ACh)}
	\end{itemize}
	
	\section{Effects of Drugs on Synaptic Transmission}
	
	\subsection{Effects of Drugs on Neurotransmission}
	
	\begin{itemize}
		\item \textbf{Ligand:} any molecule that binds to a receptor (NTs are ligands).
		\item Thus, drugs are also considered ligands. 
	\end{itemize}
	
	\subsection{Agonists vs. Antagonists}
	
	\begin{itemize}
		\item Agonist: a drug that mimics or facilitates the actions of a neurotransmitter. 
		\subitem \example{nicotine is an agonist of acetylcholine}. 
		\item Antagonist: a drug that inhibits the actions of a neurotransmitter. 
		\subitem \example{Many antipsychotic drugs are antagonists of dopamine}.
	\end{itemize}
	
	\section{Figures and Diagrams}
	
	
	
	\section{References}
	Include all references used here in standard citation format.
	
\end{document}
