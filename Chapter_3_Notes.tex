\documentclass[12pt,a4paper]{article}

% Packages
\usepackage[utf8]{inputenc}
\usepackage[margin=1in]{geometry}
\usepackage{amsmath, amssymb, amsthm}
\usepackage{graphicx}
\usepackage{float}
\usepackage{hyperref}
\usepackage{xcolor}
\usepackage{fancyhdr}
\usepackage{enumitem}

% Header and Footer
\pagestyle{fancy}
\fancyhf{}
\fancyhead[L]{Neuroscience Notes}
\fancyhead[R]{\thepage}

% Custom Commands
\newcommand{\concept}[1]{\subsection*{\textcolor{blue}{Concept: #1}}}
\newcommand{\definition}[1]{\subsubsection*{\textcolor{teal}{Definition: #1}}}
\newcommand{\example}[1]{\paragraph{\textcolor{purple}{Example:}} #1}
\newcommand{\summary}{\section*{\textcolor{red}{Summary}}}
\newenvironment{keypoints}{\paragraph{\textbf{Key Points:}}\begin{itemize}[label=--]}{\end{itemize}}

% Title
\title{\textbf{Neuroscience Notes}}
\author{}
\date{}

\begin{document}
	
	\maketitle
	\tableofcontents
	\newpage
	
	\section*{Anatomy of the Nervous System}
	
	
\begin{figure}
	\centering
	\includegraphics[width=0.7\linewidth]{"Pictures/Anatomy of nervous system/flowchart"}
	\caption{Anatomy of nervous system flowchart}
	\label{fig:flowchart}
\end{figure}

	
	\section{Divisions of the nervous systems}
	\definition{Vertebrates have two major divisions:}
	\begin{enumerate}
		\item \textbf{Central Nervous System (CNS), consists of the brain, spinal cord, and retina of the eye.}
		\item \textbf{Peripheral Nervous System} (PNS): everything outside of the brain, spinal cord, and retina.
		\subitem Within the PNS, there are two more subdivisions: 
		\subsubitem \textbf{Somantic Nervous System (SNS): interacts with our surrounding environment.}
		\subsubitem \textbf{Autonomic Nervous System (ANS): regulates our internal environment.}
		
		
\begin{figure}
	\centering
	\includegraphics[width=0.7\linewidth]{"Pictures/Anatomy of nervous system/CNS and PNS"}
	\caption{Central Nervous System }
	\label{fig:cns-and-pns}
\end{figure}
		  
		\subitem Each of the above are comprised of afferent and efferent nerves, where: 
		\begin{enumerate}
			\item \textbf{Afferent} = \textit{approach} the CNS.
			\item \textbf{Efferent} = \textit{exiting} the CNS. 
		\end{enumerate}
	\end{enumerate}
	
	\subsection{The Somatic Nervous System}
	Within the SNS: 
	\begin{enumerate}
		\item Afferent nerves carry information \textit{from} the sense organs to be the brain and spinal cord. 
		\item Efferent nerves carry motor/movement signals \textit{from} the brain \textit{to} the muscles. 
	\end{enumerate}
	

	
\begin{figure}
	\centering
	\includegraphics[width=0.7\linewidth]{"Pictures/Anatomy of nervous system/Aff vs eff nerves"}
	\caption{Afferent vs efferent nerve}
	\label{fig:aff-vs-eff-nerves}
\end{figure}

	
	\subsection{The Autonomic Nervous System}
	
	Afferent nerves carry sensory signals \textit{from} internal organs \textit{to} the CNS, while efferent nerves carry motor signals \textit{from} the CNS to smooth muscle and tissue... there are two kinds of efferent nerves: 
	\begin{enumerate}
		\item Sympathetic
		\subitem Nerves control energy resources during threatening situations. 
		\subsubitem \textit{Fight or flight} 
		\example{When you feel nervous, you often feel , that's due to your sympathetic systems trying to protect you.}
		\item Parasympathetic 
		\subitem Nerves conserve energy resources 
		\subsubitem "Rest and digest"
	\end{enumerate}
	
\begin{figure}
	\centering
	\includegraphics[width=0.7\linewidth]{"Pictures/Anatomy of nervous system/parasympathetic and sympathetic"}
	\caption{Parasympathetic and sympathetic}
	\label{fig:parasympathetic-and-sympathetic}
\end{figure}

	Both of these have the same targets within the nervous system but opposing functions. 
	
	\section*{Support Systems}
	
	\section{Support Systems}
	
	\subsection{The Meninges}
	
	There are three protective membranes surrounding the brain referred to as "meninges..."
	
	\begin{enumerate}
		\item \textbf{Dura Mater}: "hard mother," closest to skull and spine. 
		\item \textbf{Arachnoid Mater}: the middle layer that looks like a spider web; contains the "subarachnoid space" which is comprised of a high concentration of blood vessels. 
		\item \textbf{Pia Mater}: "pious mother" that adheres to the surface of the brain and spinal cord. 
	\end{enumerate}
	
	
\begin{figure}
	\centering
	\includegraphics[width=0.7\linewidth]{"Pictures/Support System/Meninges"}
	\caption{Meninges}
	\label{fig:meninges}
\end{figure}
	
	\subsection{The Ventricular System}
	
	First, within our ventricular system, we have cerebrospinal fluid: 
	\begin{itemize}
		\item CFS supports and cushions the central nervous system. 
		\item CSF is produced by the choroid plexus, which is a network of capillaries. 
	\end{itemize}
	
	
\begin{figure}
	\centering
	\includegraphics[width=0.7\linewidth]{"Pictures/Support System/fluid"}
	\caption{Cerebrospinal fluid (CSF)}
	\label{fig:fluid}
\end{figure}
	
	There are also various ventricles and canals that are a part of our ventricular system: 
	
	\underline{Cerebral ventricles}
	\begin{enumerate}
		\item Lateral ventricles 
		\item Third ventricle 
		\item Fourth ventricle 
	\end{enumerate}
	
	\underline{Central canal}: channel that runs the length of the spinal cord.
	
	 
\begin{figure}
	\centering
	\includegraphics[width=0.7\linewidth]{"Pictures/Support System/cerebral ventricles"}
	\caption{Cerebral ventricles}
	\label{fig:cerebral-ventricles}
\end{figure}
	
	\subsection{A Disorder of the Ventricular System}
	\textbf{Hydrocephalus}
	\begin{itemize}
		\item If a block of CSF occurs, this may result in a buildup of fluid in the ventricles, leading to hydrocephalus. 
		\item Caused by tumor, infection, birth defects, traumatic brain injury. 
		\item Treatment approach is to drain excess fluid and remove obstruction. 
	\end{itemize}
	
	
\begin{figure}
	\centering
	\includegraphics[width=0.7\linewidth]{"Pictures/Support System/hydrocephalus"}
	\caption{Hydrocephalus}
	\label{fig:hydrocephalus}
\end{figure}
	
	\subsection{The Blood-Brain Barrier}
	
	The brain simutaneously needs nutrients and protection from invaders:
	\begin{itemize}
		\item The BBB is formed by tightly packed walls of blood vessels, which prevents most chemicals from entering the brain. 
		\item What can cross
		\begin{enumerate}
			\item Small, uncharged molecules such as oxygen (02) and carbon dioxide (CO2).
			\item Specific fat-soluble vitamins such as Vitamins A and D. 
			\item Glucose and other amino acids pumped across by active transport systems. 
		\end{enumerate} 
		\item So, what can't cross? 
		\begin{enumerate}
			\item Most viruses and bacteria, except for rabies and herpes. 
			\item Most drugs 
			\subitem This can be a challenge, because sometimes with therapeutic development, we want drugs to be able to cross the BBB. 
			\example{L-DOPA for Parkinson's treatment}.
		\end{enumerate}
	\end{itemize}
	
	
\begin{figure}
	\centering
	\includegraphics[width=0.7\linewidth]{"Pictures/Support System/BBB"}
	\caption{The Blood-Brain Barrier}
	\label{fig:bbb}
\end{figure}
	
	\section*{Directions and Planes of the Nervous System}
	
	\section{Directions and Planes of the Nervous System}
	
	\subsection{Directions of the Vertebrate Nervous Systems}
	
	We use special terms to describes the directions or view of the brain and body in relation to the spinal cord. Here are some of our axes:
	
	
\begin{figure}
	\centering
	\includegraphics[width=0.7\linewidth]{"Pictures/Directions and Planes of Nervous System/Directions of the Vertebrate Nervous System"}
	\caption{Directions of the Vertebrate Nervous System}
	\label{fig:directions-of-the-vertebrate-nervous-system}
\end{figure}
	
	\begin{enumerate}
		\item Anterior / Posterior
		\subitem \textbf{Anterior}: also referred to "rostral", toward the nose end. 
		\subsubitem Whatever we're looking at is more so towards the nose, the front. 
		\subitem \textbf{Posterior}: also referred to as "caudal", toward the tail end. 
		\subsubitem Whatever we're looking at is more so towards the tail, the back. 
		
		
\begin{figure}
	\centering
	\includegraphics[width=0.7\linewidth]{"Pictures/Directions and Planes of Nervous System/Directions of the Vertebrate Nervous System cont."}
	\caption{Directions of the vertebrate nervous systems / anterior - posterior}
	\label{fig:directions-of-the-vertebrate-nervous-system-cont}
\end{figure}
		
		\item Dorsal / ventral
		\subitem \textbf{Dorsal}: towards the back or top of the head. 
		\subitem \textbf{Ventral}: towards the chest or bottom of the head. 
		
\begin{figure}
	\centering
	\includegraphics[width=0.7\linewidth]{"Pictures/Directions and Planes of Nervous System/Directions of the Vertebrate Nervous System cont 2."}
	\caption{Directions of the Vertebrate Nervous System / Dorsal - ventral}
	\label{fig:directions-of-the-vertebrate-nervous-system-cont-2}
\end{figure}
		
		
		\item Medial / Lateral 
		\subitem \textbf{Medial}: towards the middle of the body or brain. 
		\subitem \textbf{Lateral}: away from the middle of the body or brain. 
		
\begin{figure}
	\centering
	\includegraphics[width=0.7\linewidth]{"Pictures/Directions and Planes of Nervous System/Directions of the Vertebrate Nervous System cont 3."}
	\caption{Directions of the Vertebrate Nervous System / Medial lateral}
	\label{fig:directions-of-the-vertebrate-nervous-system-cont-3}
\end{figure}
		
		The following are only used for primates, such as humans, that walk upright...
		
		\item Superior / Inferior 
		\subitem \textbf{Superior}: towards the top of the head. 
		\subitem \textbf{Inferior}: towards the bottom of the head. 
		
\begin{figure}
	\centering
	\includegraphics[width=0.7\linewidth]{"Pictures/Directions and Planes of Nervous System/Directions of the Vertebrate Nervous System cont 4."}
	\caption{Directions of the Vertebrate Nervous System / Superior - Inferior}
	\label{fig:directions-of-the-vertebrate-nervous-system-cont-4}
\end{figure}
		
		
		\item Proximal / distal 
		\subitem \textbf{Proximal}: Closer to or towards the appendage of the body. 
		\subitem \textbf{Distal}: Away from the appendage. 
	\end{enumerate}
	
	
\begin{figure}
	\centering
	\includegraphics[width=0.7\linewidth]{"Pictures/Directions and Planes of Nervous System/Directions of the Vertebrate Nervous System cont 5."}
	\caption{Directions of the Vertebrate Nervous System / Proximal and Distal}
	\label{fig:directions-of-the-vertebrate-nervous-system-cont-5}
\end{figure}
	
	\subsection{Directional Planes of the Nervous System}
	
	
\begin{figure}
	\centering
	\includegraphics[width=0.7\linewidth]{"Pictures/Directions and Planes of Nervous System/Directions of the Vertebrate Nervous System cont 6"}
	\caption{Directions and Planes of Nervous System}
	\label{fig:directions-of-the-vertebrate-nervous-system-cont-6}
\end{figure}
	
	
	\section*{Major Structures of the Central Nervous System}
	
\begin{figure}
	\centering
	\includegraphics[width=0.7\linewidth]{"Pictures/Major structures of the CNS/summary"}
	\caption{Summary of the major brain structures}
	\label{fig:summary}
\end{figure}
	
	
	\section{Major Structures of the Central Nervous System}
	
	\subsection{The Spinal Cord}
	
	Just like the brain, the spinal cord is composed of both grey and white matter. 
	
	\begin{itemize}
		\item \textbf{Grey matter}: Cell bodies / somas of neurons.
		\subitem Appears as H shape in cross-section. 
		\item \textbf{White matter}: Myelinated axons
		\subitem Surrounds the grey matter. 
	\end{itemize}
	
		
\begin{figure}
	\centering
	\includegraphics[width=0.7\linewidth]{"Pictures/Major structures of the CNS/spinal"}
	\caption{the spinal cord}
	\label{fig:spinal}
\end{figure}

	There are \textbf{31 pairs} of \textbf{spinal nerves} that exit the spinal cord from either side. 
	
	As a reminder:
	\begin{itemize}
		\item \textbf{Afferent nerves} carry sensory information from the body. 
		\item \textbf{Efferent nerves} carry motor information to the muscles. 
	\end{itemize}
	
	The spinal cord, with the above nerve projections, is important for reflexes and more complex sensorimotor functions. 
	
	
\begin{figure}
	\centering
	\includegraphics[width=0.7\linewidth]{"Pictures/Major structures of the CNS/stuff"}
	\caption{the afferent nerves and the efferent nerves travel into the spinal cord}
	\label{fig:stuff}
\end{figure}
	
	\subsection{Divisions of the brain}
	
	Starting out with a tube like structure, then at a certain point in our development, it start to swell. 
	
	\underline{Three swellings} form during embryonic development.\\
	
	These three swellings form \textbf{five major divisions} by birth:
	
	\begin{enumerate}
		\item \underline{Hindbrain}: \textbf{myelencephalon, metencephalon}
		\item \underline{Midbrain}: \textbf{mesencephalon}
		\item \underline{Forebrain}: \textbf{diencephalon, telencephalon}
	\end{enumerate}	
	
	
\begin{figure}
	\centering
	\includegraphics[width=0.7\linewidth]{"Pictures/Major structures of the CNS/divsion"}
	\caption{Divisions of the brain}
	\label{fig:divsion}
\end{figure}
	
	\subsection{Hindbrain}
	\subsubsection{The Myelencephalon}
	
	\begin{itemize}
		\item The \textbf{medulla}
		\item Composed of tracts (bundles of axons) carrying signals between the rest of the brain and the body. 
		\item The \textbf{reticular formation}
		\subitem Controls sleep, attention, movement, and vital reflexes. 
		\item Activity is suppressed by opiate drugs (like heroin or morphine).
	\end{itemize}
	
	\subsubsection{The Metencephalon}
	\begin{itemize}
		\item Contains the \textbf{pons} and the \textbf{cerebellum}
		\subitem Sleep paralysis has something to do with abnormal activity in the pons. 
		\subitem Sensorimotor functions
		\item Also houses part of the reticular formation. 
	\end{itemize}
	
	
\begin{figure}
	\centering
	\includegraphics[width=0.7\linewidth]{"Pictures/Major structures of the CNS/structure"}
	\caption{Structures of the human myelencephalon(medulla) and metencephalon}
	\label{fig:structure}
\end{figure}
	
	\subsection{Midbrain}
	\subsubsection{The Mesencephalon}
	\begin{itemize}
		\item \textbf{Tectum} (dorsal surface or "root")
		\subitem \textbf{Superior colliculi} - processing of visual information. 
		\subitem \textbf{Inferior colliculi} - processing of auditory information. 
		\item \textbf{Tegmentum} (ventral surface or "floor")
		\subitem \textbf{Periaqueductal grey} - analgesia (pain relief)
		\subitem \textbf{Substantia nigra and red nucleus} - motor (movement) functions. 
	\end{itemize}
	
\begin{figure}
	\centering
	\includegraphics[width=0.7\linewidth]{"Pictures/Major structures of the CNS/midbrain"}
	\caption{The human mesencephalon (midbrain)}
	\label{fig:midbrain}
\end{figure}
	
	
	\subsection{Forebrain}
	\subsubsection{The Diencephalon}
	\begin{itemize}
		\item \textbf{Thalamus}: this is considered our "sensory relay station". Nuclei (subregions) of the thalamus receive information from sensory receptors and relay the signals to the cerebral cortex. 
		\item \textbf{Hypothalamus}: controls motivated behaviors (sleepng, eating, and sex).
		\subitem Regulates hormone release from the \textbf{pituitary gland}.
		\item \textbf{Optic chiasm}: the point at which the nerves from each eye come together. 
		\item \textbf{Mamillary bodies}: control aspects of memory.  
	\end{itemize}
	
	
\begin{figure}
	\centering
	\includegraphics[width=0.7\linewidth]{"Pictures/Major structures of the CNS/diencephalon"}
	\caption{The human diencephalon}
	\label{fig:diencephalon}
\end{figure}


	
\begin{figure}
	\centering
	\includegraphics[width=0.7\linewidth]{"Pictures/Major structures of the CNS/hypothalamus"}
	\caption{The human hypothalamus (in color) in relation to the optic chiasm and the pituitary gland}
	\label{fig:hypothalamus}
\end{figure}
	\subsubsection{The Telencephalon}
	\begin{enumerate}
		\item \textbf{The Cerebral Cortex}
		\begin{itemize}
			\item Comprised of a left and right hemisphere. 
			\item Hemispheres are divided by the \textbf{medial longitudinal fissure}.
			\item Hemispheres are connected by the \textbf{cerebral commissures}.
			\example{\textbf{The corpus callosum}}
			\item \textbf{Convolutions} (furrows) serve the increase surface area. 
		\end{itemize}
		
		
\begin{figure}
	\centering
	\includegraphics[width=0.7\linewidth]{"Pictures/Major structures of the CNS/fissure"}
	\caption{Medial Longitudinal Fissure}
	\label{fig:fissure}
\end{figure}

\begin{figure}
	\centering
	\includegraphics[width=0.7\linewidth]{"Pictures/Major structures of the CNS/corpus"}
	\caption{The Corpus Callosum}
	\label{fig:corpus}
\end{figure}

		
		\item \textbf{The Cerebral Cortex}: each hemisphere can be divided into four lobes: 
		\begin{itemize}
			\item \textbf{Frontal lobe} - responsible for complex cognition and some motor functions. 
			\item \textbf{Parietal lobe} - touch sensations, attention, and spatial awareness. 
			\item \textbf{Temporal lobe} - hearing, language, visual patterns, learning and memory. 
			\item \textbf{Occipital lobe} - Vision. 
		\end{itemize}
		
		
\begin{figure}
	\centering
	\includegraphics[width=0.7\linewidth]{"Pictures/Major structures of the CNS/lobe"}
	\caption{Four lobes of the human brain}
	\label{fig:lobe}
\end{figure}
		
		\item Subcortical System 
		\begin{itemize}
			\item The \textbf{limbic system} circles the thalamus. 
			\subitem Amygdala, fornix, cingulate gyrus, and hippocampus. 
			\subitem Motivated behaviors, emotion, learning and memory. 
			\item The \textbf{basal ganglia} on each side of the thalamus. 
			\subitem \textbf{Caudate nucleus} and \textbf{putamen (striatum), globus pallidus}.
			\subitem Voluntary movements.
		\end{itemize}	
	\end{enumerate}
	
\begin{figure}
	\centering
	\includegraphics[width=0.7\linewidth]{"Pictures/Major structures of the CNS/limbic"}
	\caption{The major structures of the limbic system: amygdala, hippocampus, cingulate cortex, fornix, septum, and mammillary body.}
	\label{fig:limbic}
\end{figure}


	
\begin{figure}
	\centering
	\includegraphics[width=0.7\linewidth]{"Pictures/Major structures of the CNS/more"}
	\caption{The basal ganglia: amygdala, striatum (caudate plus putamen), and globus pallidus. Notice that, in this view, the right globus pallidus is largely hidden behind the right thalamus and the left globus pallidus is totally hidden behind the left putamen. Although the globus pallidus is usually considered to e a telencephalic structure, it actually originates from diencephalic tissue that migrates into its telencephalic location during the course of prenatal development.}
	\label{fig:more}
\end{figure}
	
	\section*{Cells of the Nervous System}

	\section{Cells of the Nervous System}
	
	There are two main types of cells in the CNS:
	
	
\begin{figure}
	\centering
	\includegraphics[width=0.7\linewidth]{"Pictures/Cells of the CNS/cell8"}
	\caption{Neuron anatomy Summary}
	\label{fig:cell8}
\end{figure}
	
	\subsection{Neurons}
		\begin{itemize}
			\item These are specialized for the reception, conduction, and transmission of electrochemical signals. 
		\item The main part of the neuron are comprised of the:
		\subitem \textbf{Dendrites }
		\subitem \textbf{Cell body}
		
		\subitem \textbf{Cell membrane}
		
		\subitem \textbf{Axon hillock}
		
		\subitem \textbf{Axon - myelin sheath} and \textbf{Nodes of Ranvier}
		\subitem \textbf{Terminal buttons / presynaptic terminals/axon terminals}. 
		\end{itemize}
	
	\subsubsection{Dendrites}
	
	Dendrites are the short branches at the receiving end of the neuron (typically at the "top" of a neuron in illustrations; branch from soma)
	
	\begin{itemize}
		\item \textbf{Receptors} on the surface of dendrites receive information from other preceding neurons. 
		\item \textbf{Dendritic spines}: short outgrowths to increase surface area for signal input and receival. 
		\item The shapes of dendrites and spines vary across cells, and can vary within a cell as well based on the environment. 
		\subitem Can change due to learning, stress, and drugs. 
	\end{itemize}
	
	
\begin{figure}
	\centering
	\includegraphics[width=0.7\linewidth]{"Pictures/Cells of the CNS/cells1"}
	\caption{Schematics of a dendrite}
	\label{fig:cells1}
\end{figure}
	
	\subsubsection{The Soma, or cell body}
	
	\begin{itemize}
		\item The soma serves as the metabolic center of the neuron. 
		\item Contains: organelles such as the nucleus, mitochondria, and ribosomes. 
		\item Contains DNA. 
	\end{itemize}
	
\begin{figure}
	\centering
	\includegraphics[width=0.7\linewidth]{"Pictures/Cells of the CNS/cell2"}
	\caption{Schematic of cell body/axon}
	\label{fig:cell2}
\end{figure}
	
	
	\subsubsection{The Cell Membrane}
	
	\begin{itemize}
		\item The cell membrane of a neuron is composed of a \textbf{lipid bilayer} (two layers of \textbf{fat molecules}).
		\item Numerous different \textbf{protein molecules} are embedded within the lipid bilayer. 
	\end{itemize}
	
	
\begin{figure}
	\centering
	\includegraphics[width=0.7\linewidth]{"Pictures/Cells of the CNS/cell3"}
	\caption{Cell membrane}
	\label{fig:cell3}
\end{figure}
	
	\subsubsection{The Axon Hillock}
	
	\begin{itemize}
		\item Portion of the soma that connects to the axon. 
		\item Involved in the control of initiation of electrical impulse upon the receival of signals from the environment or other neurons. 
	\end{itemize}
	
	
\begin{figure}
	\centering
	\includegraphics[width=0.7\linewidth]{"Pictures/Cells of the CNS/cell4"}
	\caption{Axon hillock}
	\label{fig:cell4}
\end{figure}
	
	\subsubsection{The Axon}
	\begin{itemize}
	\item Long, thin fiber extending from the cell body. 
	\item The neuron's "information sender..." the axon is responsible for transmitting electrical impulses to carry to other neurons. 
	\item Many axons are insulated with a \textbf{myelin sheath}.  
	\end{itemize}
	
	
\begin{figure}
	\centering
	\includegraphics[width=0.7\linewidth]{"Pictures/Cells of the CNS/cell5"}
	\caption{Axon}
	\label{fig:cell5}
\end{figure}
	
	\subsubsection{The Myelin Sheath}
	
	\begin{itemize}
		\item Fatty white substance insulating axons. 
		\item \textbf{Propagates} and increases the speed of signal conduction down the neuron. 
		\item The myelin sheath is interrupted by the \textbf{Nodes of Ranvier}, which allows the signal to jump from node to node via \textbf{saltatory conduction}. 
	\end{itemize}
	
	
\begin{figure}
	\centering
	\includegraphics[width=0.7\linewidth]{"Pictures/Cells of the CNS/cell6"}
	\caption{Myelin Sheath}
	\label{fig:cell6}
\end{figure}
	
	\subsubsection{Axon Terminals}
	
	\begin{itemize}
		\item AKA \textbf{presynaptic terminals} or \textbf{terminal buttons}
		\item These are the endpoints of axon branches. 
		\item Site of release of chemical signals that will reach other neurons. 
	\end{itemize}
	
	
\begin{figure}
	\centering
	\includegraphics[width=0.7\linewidth]{"Pictures/Cells of the CNS/cell7"}
	\caption{Axon Terminals}
	\label{fig:cell7}
\end{figure}
	
	\subsubsection{Neuron Classification}
	
	Neurons can be classified by their shape, or differing structures based off their functions and processes...
	
	We have more Glial cells than neurons. 
	
	
\begin{figure}
	\centering
	\includegraphics[width=0.7\linewidth]{"Pictures/Cells of the CNS/cell9"}
	\caption{Neuron classification}
	\label{fig:cell9}
\end{figure}
	
	\subsection{Glial cells}
	
	\begin{itemize}
		\item Glia are non-neuronal cells that come in a variety of sizes, shapes, and types and differ in functions:
		\subitem \textbf{Microglia}
		\subitem \textbf{Oligodendrocytes and Schwann cells}
		\subitem \textbf{Astrocytes}
		\subitem \textbf{Radial Glia}
	\end{itemize}
	
	
	\subsubsection{Microglia}
	
	Microglia are small "scavenger" cells that respond to disease and injury.
	
	\begin{itemize}
		\item Engulf and destroy viruses, bacteria, and other microorganisms (phagocytosis).
		\item Go from a dormant to activated state during CNS inflammation. 
		\item Act as trash pickup for neurons. 
	\end{itemize}
	
\begin{figure}
	\centering
	\includegraphics[width=0.7\linewidth]{"Pictures/Cells of the CNS/cell11"}
	\caption{Microglia}
	\label{fig:cell11}
\end{figure}
	
	
	\subsubsection{Oligodendrocytes and Schwann Cells}
	
	\begin{itemize}
		\item \textbf{Oligodendrocytes} build the myelin sheath that wraps around cells in the CNS. 
		\item \textbf{Schwann cells} build the myelin sheath that wraps around cells in the PNS. 
	\end{itemize}
	
	
\begin{figure}
	\centering
	\includegraphics[width=0.7\linewidth]{"Pictures/Cells of the CNS/cell12"}
	\caption{Oligodendrocytes and Schwann Cells}
	\label{fig:cell12}
\end{figure}
	
	\subsubsection{Astrocytes}
	
	Astrocytes are star-shaped cells with a variety of functions:
	
	\begin{itemize}
		\item Cover outer surface of blood vessels in the brain to assist with \textbf{BBB} function. 
		\item Synchronize activity of neurons by wrapping around the \textbf{synapses}, which is the gap or connective point between neurons. 
		\item Also synthesize certain neurotransmitters (i.e., glutamate) and mediate neurotransmission. 
	\end{itemize}
	
\begin{figure}
	\centering
	\includegraphics[width=0.7\linewidth]{"Pictures/Cells of the CNS/cell13"}
	\caption{Astrocytes}
	\label{fig:cell13}
\end{figure}
	
	
	\subsubsection{Radial Glia}
	
	\begin{itemize}
		\item Guide migration and growth of neurons during brain development. 
		\item Later differentiate into neurons or other types of glial cells. 
	\end{itemize}
	
\begin{figure}
	\centering
	\includegraphics[width=0.7\linewidth]{"Pictures/Cells of the CNS/cell14"}
	\caption{Radial Glia}
	\label{fig:cell14}
\end{figure}
	
	\section{References}
	
	As mentioned in the lecture (Laura Mcauliffe, Ph.D Candidate, 2025),
	
	\begin{thebibliography}{9}
		\bibitem{LectureSlides}
		Laura Mcauliffe, \emph{The Anatomy of the Nervous System}, Biopsychology (PSYC 372), George Mason Univerity, 2025. PowerPoint presentation.

		\bibitem{LectureSlides}
		Laura Mcauliffe, \emph{Support Systems}, Biopsychology (PSYC 372), George Mason Univerity, 2025. PowerPoint presentation.

		\bibitem{LectureSlides}
		Laura Mcauliffe, \emph{The Directions and Planes of the Nervous System}, Biopsychology (PSYC 372), George Mason Univerity, 2025. PowerPoint presentation.

		\bibitem{LectureSlides}
		Laura Mcauliffe, \emph{Major Structures of the CNS}, Biopsychology (PSYC 372), George Mason Univerity, 2025. PowerPoint presentation.

		\bibitem{LectureSlides}
		Laura Mcauliffe, \emph{Cells of the CNS}, Biopsychology (PSYC 372), George Mason Univerity, 2025. PowerPoint presentation.
	\end{thebibliography}
	
\end{document}
